\documentclass[aps,pre,twocolumn
,groupedaddress]{revtex4-1}
%\documentclass[a4paper,onecolumn,12pt]{article}
\usepackage{amsmath}
 \usepackage{amsfonts}
 \usepackage{amssymb}
 \usepackage{graphicx}
 \usepackage{subfigure}
 \usepackage{enumerate}
 \usepackage{verbatim}% for \begin{comment} \end{comment}
 \usepackage{color}
 \usepackage{bm}
 \begin{document}
 \title{Interacting Classical Gas}
 \author{Zhiyu}
 \affiliation{Fudan University, Shanghai, China}
 
 
\begin{abstract}
\end{abstract}
 \maketitle
% \tableofcontents
% \newpage
\section{Introduction}


\section{Preparation}
\subsection{Scaling behaviour}
We investigated the classical gas in harmonic trap with a repulsive interaction. We chose one of the simplest types of interaction:
\begin{equation}
F=\left\lbrace
\begin{aligned}
&F=F0 	&,x\textless\sigma\\
&F=0 	&,x\textgreater\sigma
\end{aligned}
\right.
\label{eq:preparation1}
\end{equation}
so the Hamiltonian is
\begin{equation}
H=\frac{1}{2}m\omega_0^2\sum_{i}x_i^2+\frac{1}{2}m\sum_i v_i^2+\sum_{\left|x_i-x_j\right|\textless\sigma}F_0\left(\sigma-\left|x_i-x_j\right|\right)
\label{eq:preparation2}
\end{equation}
It can be shown that eq.\ref{eq:preparation2} can be rewrite into the following form:
\begin{equation}
\tilde{H}=\frac{1}{2}\sum_{i}\tilde{x_i}^2+\frac{1}{2}\sum_i\left(\frac{d\tilde{x}}{d\tilde{t}}\right)^2+\sum_{\left|\tilde{x_i}-\tilde{x_j}\right|\textless 1}F_0\left(1-\left|\tilde{x_i}-\tilde{x_j}\right|\right)
\end{equation}
with the transformation:
\begin{equation}
\left\lbrace
\begin{aligned}
&\tilde{H}&=&\frac{H}{m\omega_0^2\sigma^2}\\
&\tilde{x_i}&=&\frac{x_i}{\sigma}\\
&\tilde{t}&=&\omega_0t\\
&\tilde{F_0}&=&\frac{F_0}{m\omega_0^2\sigma}\\
\end{aligned}
\right.
\end{equation}

In this manner, we reduced number of the parameters in our model into three: $\tilde{H}$, $\tilde{F_0}$, $N$. Equivalently, we set $\sigma$, $\omega_0$ and $m$ to 1 in our numerical simulation. Throughout, we will use ``E" and ``$F_0$" to denote their reduced version  $\tilde{H}$, $\tilde{F_0}$.

\subsection{Overview}
\subsubsection{Relation between Radius and Energy}
The radius of the cloud R is defined as the root-mean-square of particles' position $\left\lbrace x_i\right\rbrace$. When E is very large, the interaction energy can be omitted, so E is dominated by the kinetic energy and potential energy of harmonic oscillators. Thus, $E=\omega_0^2\sum_{i}x_i^2=N\omega_0^2R^2$ 
\begin{figure}[hbtp]
\centering
\includegraphics[scale=0.3]{D:/OriginUserfiles/MPI_internship/E_R_fitting.jpg}
-\caption{E-R} This picture shows $E=kR^2$ when energy is large. ($k=N\omega_0^2$)
\end{figure}

\subsubsection{Ground states}
For the ground state of this system, one could predict that \\(1) When the $\sigma$ is larger than the size of the cloud (in view of the scaling behaviour, large $\sigma$ is equivalent to \textbf{small $\mathbf{F_0}$ limit} when we fix $\sigma=1$), it can be easily shown that the ground state is crystal-like, since all every particle keeps an equal distance with its nearest neighbours, which is $\frac{2F_0}{\omega_0^2}$. The low-lying excitation is particles' independent harmonic oscillation around their equilibrium position.\\(2) When \textbf{$\mathbf{F_0}$ is very large} the molecules act like a hardcore gas, while the size of particle is just $\sigma$.

In the fig.\ref{fig:GS1}, we can see this crossover from the ``solid-like" limit (left) to the ``hardcore gas" limit (right).

\begin{figure}[hbtp]
\centering
\includegraphics[scale=0.5]{D:/matlab2016/MolecularDynamics/figure/latexpic/groundstate1.eps}
\caption{particles' position at ground states }
\label{fig:GS1}
\end{figure}

\begin{comment}
\begin{figure}[hbtp]
\centering
\includegraphics[scale=0.3]{D:/matlab2016/MolecularDynamics/figure/Eg1.eps}
\caption{ground state Energy with $F_0$}
\end{figure}
\end{comment}

\newpage
\section{Breathing Frequency}
Usually, to see the response of quantum gas to a quench, one begin simulation with the ground state and then suddenly vary some parameter by a little, for instance, $\omega_0'\rightarrow\omega_0=\omega_0'+\Delta\omega$. In that case, one will observe the radius R oscillating at a certain frequency, which is usually called ``breathing mode".
However, it is quite different in the classical version. Firstly, the ground state may be meaningless. In quantum case, the reason why we care about ground state and low lying excitations is that BEC occur at low temperature. But in classical case, there is no BEC. So it is not necessary to focus on low lying excitations in classical case. Besides, since classical limit is $\bar{h}\rightarrow 0$, no matter how low the kT is, $\bar{h}/kT$ is always zero so that classical behavious of low lying excitations is quite different from the quantum one. Secondly, it is no longer necessary to vary parameters ``by a little" in the quench. In quantum case, the small quench $\omega_0'\rightarrow\omega_0=\omega_0'+\Delta\omega$ means the spectrum is shifted slightly, thus one may expect to see some beat phenomenon({\color{red}{?}}). But in classical case, since $\bar{h}$ vanishes, no matter how small the quench is, one can never recover or proximate the quantum case. Thus it is not necessary to limit ourselves to small quench. 

Furthermore, notice that X and P (normalized) are symmetric for simple harmonic oscillator. In weak interaction limit, the distribution cloud should be circular when the gas is completely thermalized. The difference between thermalized cloud before and after quench is that they are circular under different normalization of phase-space. If we look at both of them in the phase-space normalized according to the final $\omega_0$, we will find shape of the cloud before the quench is an ellipse while the final one is a circle.

Since the number of particles we use in the Molecular Dynamics simulation is limited (usually 5 to 20), the random noise is significant. If we start with a state whose configuration in phase-space is an ellipse slightly deviated from circle, the oscillating amplitude will be too small to be distinguished from noise. Therefore, we start from an extreme case -- line distribution (random x, zero p) and observe the oscillation of R(t). 
\begin{figure}[hbtp]
\center
\includegraphics[scale=0.6]{D:/matlab2016/MolecularDynamics/figure/latexpic/freq_scanF_differentE_log_2.eps}
\caption{breathing mode frequency measured at different E and $F_0$}
\label{Breathingfrequency1}
\end{figure}

Without interaction, the breathing mode frequency is exactly 2. When there is interaction, the frequency will drift to some other value near 2. We measure the radius of the cloud $R(t)$ and get the frequency spectrum of its oscillation behaviour by Fourier transform. Then we take the peak frequency near 2 as the breathing mode frequency. The frequency measured in different E and $F_0$ is shown in the Fig.\ref{Breathingfrequency1}.

The mechanism that deviate breathing frequency from 2 is simple in real space. When two particle bounce, they exchange momentum immediately. It can be interpret as particle A carries its momentum $P_A$ and jump $\sigma$ to the right, while particle B carries its momentum $P_B$ and jump $\sigma$ to the left. In this manner, every collision will save a particle some time $\frac{\sigma}{v}$ . Since $v$ is proportional to R while the number of collision each particle experiences in one period of harmonic oscillation can be estimated by N, we will get $\delta\sim N^{\frac{3}{2}}E^{-\frac{1}{2}}$. 

To see this picture better, let us first introduce a approach to simplify this problem:

\subsection{Rotating frame in Phase-space}
Traditionally, we think of there is a distribution cloud in phase space, and due to harmonic trap, the cloud will rotate at frequency $\omega_0$. When there is interaction, we may think there is a small modification of the frequency $\omega=\omega_0+\delta$. Now we want to understand how $\delta$ is dependent on parameters, so it is better to stand in the rotating frame in the phase space. In this frame, the picture we will see is as follow:\\
\begin{enumerate}[\textbf{*}]
\item All particles are stationary when there is no interaction.
\item The real X and P axis are rotating counter-clockwise, using real X axis to measure distance between particles to determine the interaction at each moment.
\item When there is interaction, particles will gain a ``velocity" in phase space: $\dot{P}=F/m$ . Other effects are all cancelled by frame rotation.
\end{enumerate}


This picture is actually a classical version of the interaction picture. With this picture, we will soon find the dynamic analysis greatly simplified.

In our system, we choose initial velocity to be zero, which means, the particles are aligned on $X|_{t=0}$ axis in phase-space at the beginning. If the breathing frequency of the system is $\omega=\omega_0+\delta$, we will expect to see a line (which may gradually deform into an oval cloud or even an isotropic cloud) rotating at $\delta$ in rotating frame. If we plot x-p with t, we expect to see a spiral motion. Since the motion we see is an additional rotation in rotating frame, we will call it precession. This precession directly leads to the $\delta$.


\begin{figure}[hbtp]
\centering
\includegraphics[scale=0.6]{D:/matlab2016/MolecularDynamics/figure/latexpic/rotatingframe_example2.eps} 
\includegraphics[scale=0.6]{D:/matlab2016/MolecularDynamics/figure/latexpic/rotatingframe_example1.eps} 
\caption{precession motion in phase-space observed in rotating frame (only show one(two) particles' trajectory in upper(lower) picture for clarity)}
\label{Breathingfrequency2}
\end{figure}


{\color{red}{ONE SET OF PICTURES HERE TO SHOW THE TIME EVOLUTION OF THE CLOUD IN PHASE-SPACE:}}
N=20, same E and F0, several initial states:
Stationary frame: T/4 T/2 3T/4 T
rotating frame: T/4 T/2 3T/4 T



\subsection{Estimating Breathing Frequency}
Let us begin with two-particle case. When interaction is very strong, the effect of interaction is equivalent to exchanging momentum when two particles approach each other at distance $\sigma$. In the rotating frame in phase-space, this process can be interpret as follows:

\begin{figure}[hbtp]
\centering
\includegraphics[scale=0.4]{D:/matlab2016/MolecularDynamics/figure/latexpic/collision.png}
\caption{a schematic diagram of the collision process}
\label{Breathingfrequency3}
\end{figure}


Two particles flip to another line(thick gray line) which deviate a small angle $\alpha$ away from original configuration. Similar process happen when there are more particles. In every period of harmonic oscillation, each particle meet N particles and collide 2N times, while half of the collisions (N times) are between particles with huge difference in momentum, which is just the process shown in figure\ref{Breathingfrequency3}. (As for the remaining half of collisions, colliding particles have small difference in their momentum. Let us say, both particles have positive momentum. In fig.\ref{Breathingfrequency3}, it can be shown that two particles on the same half of the P-axis cannot have strong effect on the rotation of distribution.)

Let us estimate the precession angle $\alpha$ with $\frac{\sigma}{R}$, which follows from the two-particle analysis. Radius of the cloud R can be estimated according to $E=Nm\omega_0^2R^2$
We will get the precession angular velocity 
\begin{equation}
\delta=2\omega_0\frac{N^\frac{3}{2}m^{\frac{1}{2}}E^{-\frac{1}{2}}\sigma\omega_0}{2\pi}
\label{eq:breathingfrequency1}
\end{equation}
\begin{figure}[hbtp]
\begin{center}
\includegraphics[scale=0.7]{D:/matlab2016/MolecularDynamics/figure/latexpic/N=5_freq_at_F0=10000_fitting.eps}
\caption{Frequency measured at N=5} 
Since the prediction is only an estimation by order, which means there could be some extra effective coefficient before the R. In view of this, the result is satisfying.

\includegraphics[scale=0.4]{D:/matlab2016/MolecularDynamics/figure/latexpic/N=5_freq_at_F0=10000_sup1.eps}
\includegraphics[scale=0.4]{D:/matlab2016/MolecularDynamics/figure/latexpic/N=5_freq_at_F0=10000_sup2.eps}
\caption{Spectrum at the beginning and the end of measurement}
\end{center}
 Two spectrum above show the error of our frequency measurement. The line width of the peak is significant compared with the $\delta$. The peak width comes from FT in finite time.
\end{figure}


\begin{figure}[hbtp]

\centering
\includegraphics[scale=0.7]{D:/matlab2016/MolecularDynamics/figure/latexpic/N=20_freq_at_F0=10000_2.eps}
\caption{N=20}
\includegraphics[scale=0.4]{D:/matlab2016/MolecularDynamics/figure/latexpic/N=20_freq_at_F0=10000_2_sup1.eps}
\includegraphics[scale=0.4]{D:/matlab2016/MolecularDynamics/figure/latexpic/N=20_freq_at_F0=10000_2_sup2.eps}
\end{figure}

When F0 is not that large, the frequency behaviour seems complicated (see fig.\ref{Breathingfrequency1}). Usually, when F0 increase, frequency goes down below 2 first and then rise up and converge to some value higher than 2. The reason could also be well understood with the mechanism described before. In the former discussion, we goes to high F0 limit, which means particles exchange momentum in infinitesimal time. For F0 is not very big, the finite interaction time should be taken into consideration. During this time, the behaviour of each particle in phase space is exactly moving along the direction of the real P axis at 'velocity' $F0/m$. Since the real P axis is rotating counter-clockwise, the particle will follow the circular trajectory(orange and green dashed line in fig.\ref{Breathingfrequency3}) the $\alpha$ that we once use to estimate the precession motion should decrease with the increase of interaction time. There should be some ``critical point" where $\alpha$ goes from plus to minus (dashed green line). At this point, we will see the precession stop so that the frequency is 2. 
 

\begin{figure}[hbtp]
\centering
\includegraphics[scale=0.25]{D:/matlab2016/MolecularDynamics/figure/latexpic/N=20_freq_at_differentF0_scanF0_1.eps}
\caption{•}
\end{figure}

 

Before completing the discussion, it is necessary to point out that none of the argument here require ``energy thermalization". In addition, if the system is completely thermalized in terms of their phase of orbit in phase space, the argument will no longer work -- there is a stable isotropic distribution in phase space, no matter how the cloud rotate, no oscillation can be observed.


\section{Thermalization}
\subsection{Dynamics Study}
It is natural to think that a many-body system with interaction will be thermalized in ``usual" case. To study the condition of thermalization is equivalent to find out the mechanism that prohibits the system from ergodic. In view of this, we begin the study with two-particle motion. 
\subsubsection{two particles}

\begin{figure}[hbtp]
\centering
\includegraphics[scale=0.2]{C:/Users/admin/Desktop/test.png}
\caption{effective potential for two-particle system}
\label{fig:thermalization2}
\end{figure}

To see the way interaction change the orbit (energy level) of the particles, we had better start from two-particle case.

We can always think of them in the reference frame of their center of mass ``C". In frame C, one can easily show that when there is no interaction what we will see is that two particles are bound together by a harmonic trap center at C (black line in fig.\ref{fig:thermalization2}). So their relative motion is also a harmonic oscillation. 

In this case, the system has two frequency components. The first one is the frequency of C, which is just $\omega_0$ . The other one is the frequency of their relative motion $\omega_r$, which is slightly deviated from $\omega_0$. The deviation grows with the increase of $F_0$ and the decrease of internal energy of the pair. As a result, when the $F_0$ is small and when the E is large, a beat with low frequency $\omega_r-\omega_0=\delta$ will exist. This phenomenon means energy transfer between particles are inefficient, thus it will slow down the thermalization process. 



\begin{figure*}
\begin{center}

\subfigure[$E_i=2$]{
\begin{minipage}[b]{0.2\textwidth}
\includegraphics[scale=0.2]{D:/matlab2016/MolecularDynamics/figure/latexpic/twoparticle_Er=2_energy.eps}
\includegraphics[scale=0.2]{D:/matlab2016/MolecularDynamics/figure/latexpic/twoparticle_Er=2_phasespace.eps}
\end{minipage}
}
\subfigure[$E_i=3$]{
\begin{minipage}[b]{0.2\textwidth}
\includegraphics[scale=0.2]{D:/matlab2016/MolecularDynamics/figure/latexpic/twoparticle_Er=3_energy.eps}
\includegraphics[scale=0.2]{D:/matlab2016/MolecularDynamics/figure/latexpic/twoparticle_Er=3_phasespace.eps}
\end{minipage}
}
\subfigure[$E_i=5$]{
\begin{minipage}[b]{0.2\textwidth}
\includegraphics[scale=0.2]{D:/matlab2016/MolecularDynamics/figure/latexpic/twoparticle_Er=5_energy.eps}
\includegraphics[scale=0.2]{D:/matlab2016/MolecularDynamics/figure/latexpic/twoparticle_Er=5_phasespace.eps}
\end{minipage}
}
\subfigure[$E_i=10$]{
\begin{minipage}[b]{0.2\textwidth}
\includegraphics[scale=0.2]{D:/matlab2016/MolecularDynamics/figure/latexpic/twoparticle_Er=10_energy.eps}
\includegraphics[scale=0.2]{D:/matlab2016/MolecularDynamics/figure/latexpic/twoparticle_Er=10_phasespace.eps}
\end{minipage}
}
\end{center}
\caption{the dependence of beat frequency on internal energy\label{fig:thermalization1}} the upper diagram is the energy of every particle, the lower diagram is the orbit in phase-space, where we see quasi-periodic motion clearly. Here, $E_i$ is large(compare to $F_0\sigma$). In this case, the larger $E_i$ is, the lower the beat frequency is. Actually in very low energy, we can retrieve a beat, but that is for different reason -- it is a beat between $\omega_0$ and $\frac{\omega_0}{2}+\delta $

\end{figure*}




\subsubsection{More particles}
Now let's consider the three-particle case. For two particle, the motion is non-chaotic. On the other hand, intuitively, we will say that three-body motion is chaotic so that the system could be ``thermalized" soon. However, in some cases, the time scale of thermalizing could still be very long. Suppose two of them, say, A and B, has small internal energy, which means their distance and relative velocity are both small. Meanwhile, suppose particle C has some energy quite different from A \& B. In this case, A \& B will often be in the interaction, while C will pass both of them at a high speed in each period. How will energy transfer between C and the two-particle system A and B? Since the relative velocity of C and the two-particle system is usually large, C will pass A-B pair in a short time $\tau\sim\frac{\sigma}{R}$ ($\tau\ll\frac{2\pi}{\omega}$). C gives A a push when approaching it, and then push A back when leaving.  As has been discussed in the two-particle case, this process is equivalent to giving A a very small speed($\sim O(\frac{\sigma}{R})$) in the background of a trap. Since both the position and velocity of A and B are close($|x_A-x_B|\sim\sigma$), the velocity increase of A \& B are almost the same (difference$\sim O(\frac{\sigma^2}{R^2})$). In this manner, the passing of particle C only kicks the center of mass of the A-B pair slightly, leaving the internal motion of the pair unaffected. In another word, the existence of C will not have significant effect on the energy transfer between A and B, but only ``dance" with their center of mass. The physics of the ``dance" is similar to the dance between two particle. As is shown in fig.\ref{fig:thermalization4}, the particles with energy close to each other tends to form a pair with small internal energy and the pair's internal energy transfer is relatively stable. %We only need to substitute $m$, $F_0$ \& $\sigma$ with their effective counterparts, e.g. change $m$ to $\mu$ , which is the reduced mass of m \& 2m.

\begin{figure}[hbtp]
\centering
\includegraphics[scale=0.35]{D:/matlab2016/MolecularDynamics/figure/latexpic/pair1.eps}
\caption{evolution of single-particle energy}
\label{fig:thermalization4}
\end{figure}

The argument above still holds in many-particle case. Once we start from some configuration where n particles have a set of close energy levels $\left\lbrace E_n\right\rbrace $ and another m particles have another set of energy levels close to each other $\left\lbrace E_m\right\rbrace $, and $\left\lbrace E_n\right\rbrace $ is quite different from $\left\lbrace E_n\right\rbrace $, we will find these two systems oscillating on the energy level diagram at low frequency. 
The discussion above gives us some pictures about the non-ergodic state. For these state, life time is rather long so that once the system reach such configuration (or certain energy distribution), it will take a long time(more than $10^4$ periods for N=5 case) to decay. 

\subsection{Thermalization condition}
The main obstruct to get an thermalized state (which can be examined by observing distribution) is the low frequency oscillation mentioned before. Because once these modes are excited, the relaxing time could be very long($\textgreater 10^4$). As a result, to achieve ergodic state, we have to avoid such low frequency oscillation. According to our discussion before, the solution is to make internal energy of each two-particle pair not ``too large". Though it is impossible to express the internal energy of every pair in terms of the total energy $E$, we can estimate it by the average energy, which is $\frac{E}{N}$. At least they are of the same order. The thermalization condition could be given by:
$F_0\sigma\sim\frac{E}{N}$


\begin{figure}[hbtp]
\centering
\includegraphics[scale=0.3]{D:/matlab2016/MolecularDynamics/figure/latexpic/distribution1_scanF0_E=100_3.eps}
\caption{Distribution $scan F_0, E=100, N=5, \sigma=1$}
\label{fig:thermalization5}
\includegraphics[scale=0.3]{D:/matlab2016/MolecularDynamics/figure/latexpic/distribution1_scanF0_E=100_3_freq.eps}
\caption{Frequency scan $F_0$, $E=100$, $N=5$, $\sigma=1$}

\end{figure}




\begin{figure}
\centering
\includegraphics[scale=0.3]{D:/matlab2016/MolecularDynamics/figure/latexpic/energyleveljump_fiveparticle_F0=2_1.eps}
\caption{energy of each particle, $F_0=2$}
\label{fig:thermalization6}

\centering
\includegraphics[scale=0.3]{D:/matlab2016/MolecularDynamics/figure/latexpic/energyleveljump_fiveparticle_F0=10_1.eps}
\caption{energy of each particle, $F_0=10$}
\label{fig:thermalization7}
\end{figure}

As is shown in fig.\ref{fig:thermalization5} above, the critical point for reaching Boltzmann distribution is $F_0=5$ (we always set $\sigma=1$), while the average energy $\sim 10$, as we expected, they are of the same order. As a supplementary proof of our argument, fig.\ref{fig:thermalization6} and fig.\ref{fig:thermalization7} show the great difference of single-particle energy between $F_0=2$ $\&$ $F_0=10$ at $E=100$, i.e. in the former one there is always some particle maintained at high excitation, while in the latter the system probably goes to ergodic. To date, we have verified that the threshold of thermalization, which is $\frac{NF_0\sigma}{E}\sim1 $ , and corresponding feature of their energy distribution in two side of the threshold.

One will say that this condition only means that the energy transfer in every collision is much smaller than the energy interval. It seems that these two regime are not so distinct. It is necessary to emphasis that the essential difference between these two regime lies in whether the energy transfer in every collision is correlated: In thermalizable regime, the energy transfer in every collision is much smaller than the energy interval, which definitly slows down the energy transfer. What is more, because of this, the particles is less interfered by other particles so that two-particle analysis survives. It means that for a particle pair, the energy transferred in this collision is correlated with the one in the next time! In this manner, each particle in the pair will pick up some energy in several collisions, and then return it back to its partner -- which is just what the long term beat effect tells us. Thus the particles energy is localized in certain value. On the contrary, if we go to non-thermalizable regime, the energy transfer is so fast that for each particle don't remember their partners. As a result, the energy transferred in every collision between two particles are no longer correlated, which is equivalent to say that two-particle analysis breaks down. Instead, the energy transfer is completely random, with a non-zero amplitude. On the single-particle energy diagram, what we expected to see is a ``one dimensional random walk" ,so energy levels quickly spread around the diagram.

To sum up, the thermalization threshold not only tells us whether the amplitude of energy transfer in every collision is small enough compared to particles' energy, but also shows whether the ``correlation time" of particle pair is much longer than the characteristic time of the system (the oscillation period).  




\newpage
\subsection{Verifying Boltzmann distribution}
Thermalization could have different definition. In our system, the ``thermalization" we want to find means ``losing all the memory of initial state". One of the most important information of initial state is energy distribution. We can check whether a system has gone to equilibrium by measuring its energy distribution.

 
\subsubsection{energy distribution}
For an isothermal system, the energy distribution of the whole system follows the Boltzmann distribution in equilibrium state. However, to our knowledge, there is not any obvious conclusion about the distribution for an isolated system where the total energy is conserved.

But intuitively, one will expect that if we measure the energy of a subsystem , e.g. one particle, then we will get a Boltzmann distribution because the rest part of the system serves as a bath for this particle. The temperature of this isolated system is defined according to $Nk_BT=E$. It is evident that this argument only hold when the energy of the single particle is not too big --- if one particle take up 50\% of the total energy, the rest could no longer be thought of as a good bath.

In the first picture below, we verify this argument by measuring the energy distribution of every particle at N=10 for different F0 at E=1000 (parameters here satisfy thermalizing condition). In the other two, we tested different N.

\newpage
\begin{figure}[hbtp]
\centering
\includegraphics[scale=0.5]{D:/matlab2016/MolecularDynamics/figure/latexpic/distribution_at_N=5_F0=10000.eps}
\caption{Energy distribution, N=5, F0=10000}
\label{fig:thermalization8}

\includegraphics[scale=0.47]{D:/matlab2016/MolecularDynamics/figure/latexpic/distribution_at_N=10_E=1000_edit.eps}
\caption{Energy distribution, N=10, E=1000}
\label{fig:thermalization9}

\includegraphics[scale=0.52]{D:/matlab2016/MolecularDynamics/figure/latexpic/distribution2_at_N=20_F0=10000.eps} 
\caption{Energy distribution, N=20, F0=10000}
\label{fig:thermalization10}
\end{figure}

\newpage
\begin{flushleft}
To take the first one as an example, Y axis is logarithmically scaled. X axis is the ratio between single particle energy (where half of interaction is taken into account) and total energy of the system. The blue dotted line is the predicted Boltzmann distribution (normalized according to counts). Notice that normalizing of the Boltzmann line only change the intercept, leaving slope untouched, which means the slope is the only thing to compare. In  Energy ratio smaller than 30\%, the experiment (colourful curves) fits well with the prediction. When energy of a single particle is higher, one can think of the "bath" formed by the remaining part has lower temperature, thus counts decrease faster to zero than Boltzmann law.

It is evident form $Nk_BT=E$ that the slope of Boltzmann distribution in this figure only depend on N. That is the reason why we want to measure at different N below. 

In fig.\ref{fig:thermalization9} and fig.\ref{fig:thermalization10}, the curve is slightly deviated from the standard Boltzmann distribution, which is probably due to the contribution of density of states. Possibility is proportional to Boltzmann exponential factor multiplied by density of states. The reason why we did not take DOS into account previously lies in that the DOS is a constant for simple harmonic oscillator. However, for simple harmonic oscillators with short range interaction, the energy shell is deformed in the region of $|x_i-x_j|<\sigma$ so that the DOS is no longer a constant. But since $\sigma<<R$, one can think of this as a small modification. Therefore, the tendency of the curve in fig.\ref{fig:thermalization9} and fig.\ref{fig:thermalization10} is deviated slightly from Boltzmann distribution.
 
\end{flushleft}



\newpage
\section{Relaxation in Phase-space}
To date, we have verified that the energy distribution will go to Boltzmann distribution. However, even if a system have reached Boltzmann distribution, it doesn't mean that the system has completely ``forgotten" its initial state. There is other information that could still exist, for instance, the shape of distribution cloud in phase-space. One of the reasons why we care about this quantity is that it has a strong effect on breathing behaviour. For instance, if the distribution cloud is a thin ellipse (all particles oscillate in phase), the amplitude of oscillation of the cloud radius will be very large. On the contrary, if the cloud is a circle (particles phase of oscillation are completely random), we will not see oscillation at all. Going to equilibrium not only require the thermalization mentioned in last section, but also require reaching a stable configuration in phase-space distribution (relaxation).


\subsubsection{Shape of Distribution in Phase-space}
If the oscillating phases of particles are highly correlated, for example, initial state with zero velocity, the shape of cloud will be a thin ellipse. If the phase of oscillation is completely random, the cloud should be isotropic in phase space. In view of this, we defined a shape factor S:
\\$S=\frac{a-b}{a+b}$, where a and b are long axis and short axis of inertia ellipse in phase space respectively. a,b can be calculated by diagonalizing the inertia matrices I: $I_{xx}=\sum{x^2}, I_{xp}=I_{px}=\sum{xp},I_{pp}=\sum{p^2}$
S=1 for line-shape distribution, S=0 for circular distribution.
The time evolution of S will probably show how fast the distribution ``forget" its initial shape.  


\begin{figure}[hbtp]

\centering
\includegraphics[scale=0.3]{D:/matlab2016/MolecularDynamics/figure/latexpic/shapefactor_t=4E4_N=5.eps}
\caption{N=5,F0=10000,E=1000, initial S=1}
\end{figure}
\begin{figure}[hbtp]
\centering
\includegraphics[scale=0.3]{D:/matlab2016/MolecularDynamics/figure/latexpic/shapefactor_4E4_N=5_zoomin1.eps} 
\caption{Zoom in at the beginning part}
\end{figure}
\begin{figure}[hbtp]
\centering
\includegraphics[scale=0.3]{D:/matlab2016/MolecularDynamics/figure/latexpic/shapefactor_t=4E4_N=5_average.eps}
\caption{average over every 30s to see the main tendency}
\label{fig:relaxationPS1}
\end{figure}


From fig.\ref{fig:relaxationPS1}, no evidence shows the shape of cloud will converge. Maybe the fluctuation of S is so large that make it hard to identify the convergence of S. The only way to solve this problem is using more particle simulation to suppress the fluctuation.

\begin{figure}
\centering
\includegraphics[scale=0.4]{D:/matlab2016/MolecularDynamics/figure/latexpic/shapefactor_t=5E3_N=20_F1000_E3000.eps} 
\includegraphics[scale=0.4]{D:/matlab2016/MolecularDynamics/figure/latexpic/shapefactor_t=5E3_N=20_F1000_E3000_average.eps}
\includegraphics[scale=0.4]{D:/matlab2016/MolecularDynamics/figure/latexpic/shapefactor_t=5E3_N=20_F1000_E3000_sup.eps}
\caption{N=20, F0=1000, E=3000, initial S=1}
\begin{flushleft}
Different from N=5 case, this time the fluctuation of S is not that large. In return, with an average over every 30s, the S only fluctuate between 0.5 and 0.2
\end{flushleft}
\label{fig:relaxationPS2}
\end{figure}
\begin{figure}
\centering
\includegraphics[scale=0.4]{D:/matlab2016/MolecularDynamics/figure/latexpic/shapefactor_t=5E3_N=20_F1000_E10000.eps} 
\includegraphics[scale=0.4]{D:/matlab2016/MolecularDynamics/figure/latexpic/shapefactor_t=5E3_N=20_F1000_E10000_average.eps}
\includegraphics[scale=0.4]{D:/matlab2016/MolecularDynamics/figure/latexpic/shapefactor_t=5E3_N=20_F1000_E10000_sup.eps}
\caption{N=20, F0=1000, E=10000, initial S=1}
\label{fig:relaxationPS3}
\end{figure}

For N=20(fig.\ref{fig:relaxationPS3}), the S start evolving from 1 will soon decay and start fluctuating around some lower value(0.1-0.5, perhaps dependent on temperature). To tell whether this value correspond to a stable state of the system, we can set the initial state S to be other value (e.g. zero) to see whether it will come back to the same stable value(see fig.\ref{fig:relaxationPS4}).

\begin{figure}
\centering
\includegraphics[scale=0.4]{D:/matlab2016/MolecularDynamics/figure/latexpic/shapefactor_t=5E3_N=20_F1000_E10000_S=0.eps} 
\includegraphics[scale=0.4]{D:/matlab2016/MolecularDynamics/figure/latexpic/shapefactor_t=5E3_N=20_F1000_E10000_S=0_average.eps}
\includegraphics[scale=0.4]{D:/matlab2016/MolecularDynamics/figure/latexpic/shapefactor_t=5E3_N=20_F1000_E10000_S=0_sup.eps}
\caption{N=20, F0=1000, E=10000, Initial S=0.05}
\label{fig:relaxationPS4}
\end{figure}

We may zoom in to understand how the distribution cloud is deforming. 
in fig.\ref{fig:relaxationPS5}, we can see some peaks and dips. The peaks are generated by particle ``bumping" (collision process discribed in Breathing frequency, see fig.\ref{Breathingfrequency2}) along the short axis of the cloud, while the dips are generated by ``bumping" along the long axis. Since these bumping only means momentum exchange, after a whole bumping process, the distribution should almost come back. So these peaks and dips will not affect the plateau value, as is shown in fig.\ref{fig:relaxationPS5}
\begin{figure}[hbtp]
\centering
\includegraphics[scale=0.4]{D:/matlab2016/MolecularDynamics/figure/latexpic/shapefactor_t=5E3_N=20_F1000_E10000_S=0_zoom1.eps}
\includegraphics[scale=0.4]{D:/matlab2016/MolecularDynamics/figure/latexpic/shapefactor_t=5E3_N=20_F1000_E10000_S=0_zoom2.eps} 
\includegraphics[scale=0.4]{D:/matlab2016/MolecularDynamics/figure/latexpic/shapefactor_t=5E3_N=20_F1000_E10000_S=0_zoom3.eps} 
\includegraphics[scale=0.4]{D:/matlab2016/MolecularDynamics/figure/latexpic/shapefactor_t=5E3_N=20_F1000_E10000_S=0_zoom4.eps} 
\caption{t=5E3, N=20, F=1000, E=10000, initial S=0, zoom in}
\label{fig:relaxationPS5}
\end{figure}

In view of this, we take average of a short time to clear this peaks and dips in order to see the time evolution of the overall value of S (the second picture in fig.\ref{fig:relaxationPS2},\ref{fig:relaxationPS3},\ref{fig:relaxationPS4}).

However, there is still no sufficient evidence of convergence for the overall value of S for N=20. 
\begin{figure}[hbtp]

\centering
\includegraphics[scale=0.8]{D:/matlab2016/MolecularDynamics/figure/latexpic/shapefactor_t=1500_N=50_F10000_E20000_S=1.eps}
\includegraphics[scale=0.8]{D:/matlab2016/MolecularDynamics/figure/latexpic/shapefactor_t=1500_N=50_F10000_E20000_S=1_average.eps} 
\caption{N=50, E=20000, F0=10000 ,t=1500}
\label{fig:relaxationPS6}
\end{figure}

\newpage
\subsection{Correspondence between energy distribution, behaviour of shape-factor and oscillation of R(t)}
\begin{figure}[hbtp]
\begin{center}
\subfigure[$E=3E5$]{
\begin{minipage}[n]{0.4\textwidth}
\includegraphics[scale=0.4]{D:/matlab2016/MolecularDynamics/figure/latexpic/example_Energy_relaxed_N=20_F0=1E4_E=3E5.eps}
\includegraphics[scale=0.4]{D:/matlab2016/MolecularDynamics/figure/latexpic/example_single-particle-Energy_relaxed_N=20_F0=1E4_E=3E5.eps}
\includegraphics[scale=0.4]{D:/matlab2016/MolecularDynamics/figure/latexpic/example_Shapeav_relaxed_N=20_F0=1E4_E=3E5.eps}   
\end{minipage}
}
\subfigure[$E=2E5$]{
\begin{minipage}[n]{0.4\textwidth}
\includegraphics[scale=0.4]{D:/matlab2016/MolecularDynamics/figure/latexpic/example_Energy_relaxed_N=20_F0=1E4_E=2E5.eps}
\includegraphics[scale=0.4]{D:/matlab2016/MolecularDynamics/figure/latexpic/example_single-particle-Energy_relaxed_N=20_F0=1E4_E=2E5.eps}
\includegraphics[scale=0.4]{D:/matlab2016/MolecularDynamics/figure/latexpic/example_Shapeav_relaxed_N=20_F0=1E4_E=2E5.eps}   
\end{minipage}
}
\subfigure[$E=1E5$]{
\begin{minipage}[n]{0.4\textwidth}
\includegraphics[scale=0.4]{D:/matlab2016/MolecularDynamics/figure/latexpic/example_Energy_relaxed_N=20_F0=1E4_E=1E5.eps}

\includegraphics[scale=0.4]{D:/matlab2016/MolecularDynamics/figure/latexpic/example_Shapeav_relaxed_N=20_F0=1E4_E=1E5.eps}   
\end{minipage}
}
\subfigure[$E=5E4$]{
\begin{minipage}[n]{0.4\textwidth}

\includegraphics[scale=0.4]{D:/matlab2016/MolecularDynamics/figure/latexpic/example_Shapeav_relaxed_N=20_F0=1E4_E=5E4.eps}   
\end{minipage}
}
\end{center}
\end{figure}

\newpage
\subsection{Lyapunov Exponents(unfinished)}
Lyapunov exponents(LE) describes how fast one orbit diverge from its nearby orbits in phase-space. The largest Lyapunov exponents (LLE) reflects  the time scale that system lose its memory in phase-space. We will measure the LLEs along our trajectory and plot its distribution.

\begin{figure}[hbtp]
\centering
\includegraphics[scale=0.5]{D:/matlab2016/MolecularDynamics/figure/latexpic/C_N=5_LLEevolution_F0=1000_E=3000_2.eps}
\includegraphics[scale=0.5]{D:/matlab2016/MolecularDynamics/figure/latexpic/C_N=5_Energy_F0=1000_E=3000_2.eps} 
\caption{N=5, F0=1000, E=3000, Gaussian interaction}
\label{fig:LLEexample1}
\end{figure}
\begin{figure}[hbtp]
\centering
\includegraphics[scale=0.5]{D:/matlab2016/MolecularDynamics/figure/latexpic/C_N=5_LLEevolution_F0=1000_E=10000_1.eps}
\caption{N=5, F0=1000, E=10000, Gaussian interaction}
\label{fig:LLEexample2}
\end{figure}
\begin{figure}[hbtp]
\centering
\includegraphics[scale=0.5]{D:/matlab2016/MolecularDynamics/figure/latexpic/C_N=5_LLEevolution_F0=1000_E=100000_1.eps}
\caption{N=5, F0=1000, E=100000, Gaussian interaction}
\label{fig:LLEexample3}
\end{figure}

Fig.\ref{fig:LLEexample1}, fig.\ref{fig:LLEexample2} and fig.\ref{fig:LLEexample3} serve as examples showing the way I measure LLE: I integrate and average the LLE along the trajectory, usually over 100-1000 time unit, the LLE converge to certain stable value, and then I start another LLE measurement. \textbf{From the examples above, one may judge how good the convergence is and how big the error of LLE measurement can be.}


\begin{figure}[hbtp]
\centering
\includegraphics[scale=0.8]{D:/matlab2016/MolecularDynamics/figure/latexpic/LLEdistribution_scanE_9_27.eps}
\caption{LLE distribution for N=5, F0=1000, scan E. (It seems that MATLAB has difficulty changing .fig into .eps files. To get rid of the annoying white lines you may open this figure in MATLAB with .fig format.)}
\label{fig:LLEdistribution1}
\end{figure}


fig.\ref{fig:LLEdistribution1} shows the distribution of LLE by color. The LLE decrease with the growth of Energy. The crucial part is near E=5000. The most probable value of LLE goes down from 1 to 0.1,which indicate that \textbf{the shortest time scale becomes longer than an oscillation period when E is larger than 5000}. And E=5000 is exactly our prediction value of thermalization threshold. So there is a correspondence between energy thermalization and LLE. 
The size of bin(resolution of LLE) in fig.\ref{fig:LLEdistribution1} is 0.01, so when E goes far beyond the thermalization threshold, the LLE distribution is not that clear. We only know that LLE is concentrated close to zero in high E regime.

 




\section{others(those I haven't organized it into the text in logical order but may be useful)}

\subsection{Evolution of $R(t)$ and $\sqrt{\delta^2R(t)}/R(t)$} 
Results are shown in fig.\ref{fig:others1}

\begin{figure}[hbtp]

\centering
\includegraphics[scale=0.3]{D:/matlab2016/MolecularDynamics/figure/fluc_of_R_with_time2.eps}
\caption{fluctuation of R with time }
sigma=0.01(short range), $F_0=0.1(black), 1(blue), 10(green) ,100(red)$
\label{fig:others1}
\end{figure}


\subsection{Detailed discussion of two-particle motion}
{\color{red}{This chapter previously follow the subsection ``two-particle case" in subsection ``Dynamics study" in section ``Thermalization". But then I found we haven't used these result. So I put it at the end for the time being}}.
  

From now on, let's stand in the the frame of center of mass to observe the motion of two-particle system. In this frame, without interaction, we will find two particles sitting symmetrically in an equivalent harmonic trap (black curve), whose shape is identical to the trap we see in the static frame. As a result, there frequency of relative motion is exactly $2\omega$(since two particle are identical). We label the position of center of mass as 0. And we may only analyse the motion on positive axis since positive and negative are symmetric. 




Now, let's consider the interaction between them. Because of interaction, the potential curve between $+-\frac{\sigma}{2}$ should be modified to another parabolic curve that simultaneously satisfies the following two condition:\\
1) intersect with black curve at $+-\frac{\sigma}{2}$\\
2) minima is $+-\frac{F_0}{m\omega_0^2}$, which comes from a SHO with an extra constant force $F_0$\\
Thus, it is obvious that by comparing $\frac{F_0}{m\omega_0^2}$ and $\sigma$, we can at least divide our parameter into several regime:\\
1) When $\frac{F_0}{m\omega_0^2\sigma}>0.5$, we will have the blue curve (see figure \ref{fig:thermalization2})\\
2) When $\frac{F_0}{m\omega_0^2\sigma}<0.5$, the red curve\\
3) When $\frac{F_0}{m\omega_0^2\sigma}=0.5$, the green one.\\
For those three types of curve, the significant difference is only at low energy case:\\
For the red one, particle will oscillate at constant frequency $\omega$;
For the blue one, when energy is lower, since the first derivative of the curve in the neighbourhood is non-zero, the frequency goes to infinity;
For the green curve, in low energy limit, it takes finite time to finish half period of oscillation, while infinitesimal time is spent on the slope part. Thus, frequency will converge to  $2\omega$.

\begin{figure}
\subfigure[]{
\begin{minipage}[b]{0.3\textwidth}
\includegraphics[scale=0.17]{D:/matlab2016/MolecularDynamics/figure/latexpic/freq_scanE_1.png} 
\end{minipage}
}
\subfigure[]{
\begin{minipage}[b]{0.3\textwidth}
\includegraphics[scale=0.17]{D:/matlab2016/MolecularDynamics/figure/latexpic/freq_scanE_2.png} 
\end{minipage}
}
\subfigure[]{
\begin{minipage}[b]{0.3\textwidth}
\includegraphics[scale=0.17]{D:/matlab2016/MolecularDynamics/figure/latexpic/freq_scanE_3.png} 
\end{minipage}
}
\centering
\caption{Frequency of relative motion with internal energy in three regime} Numbers in the legend is the value of $\frac{F_0}{m\omega_0^2\sigma}$
\label{fig:thermalization3}
\end{figure}



 
 \end{document}
